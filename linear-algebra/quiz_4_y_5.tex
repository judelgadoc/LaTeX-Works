\documentclass[12pt]{article}
\usepackage{authblk}
\usepackage[spanish]{babel}
\usepackage{amsmath, amssymb, amsthm}
\usepackage[utf8]{inputenc}

%% Para poner el blacksquare en demostraciones
\renewcommand{\qedsymbol}{$\blacksquare$} 

 %% Para matrices aumentadas
\makeatletter
\renewcommand*\env@matrix[1][*\c@MaxMatrixCols c]{%
  \hskip -\arraycolsep
  \let\@ifnextchar\new@ifnextchar
  \array{#1}}
\makeatother

\title{Quiz 4 y 5}
\author{Juan Pablo Delgado Cárcamo}
\affil{
 Facultad de Ingeniería \\
 Departamento de Ingeniería de Sistemas e Industrial \\
 Universidad Nacional de Colombia
}
\date{\today}

\begin{document}

\maketitle

\noindent
1. Demuestre que el conjunto
$$B = \{ x^2 + x; x - 1; x +1 \}, \ u = 3x^2 - x + 2$$
es una base del espacio $V = P_{2}$ y encuentre las coordenadas del vector $u$ respecto a la base $B \ ([u]_{B})$.
\bigskip

$$B = \{ \begin{pmatrix} 1\\ 1\\ 0 \end{pmatrix}, \begin{pmatrix} 0\\ 1\\ -1 \end{pmatrix}, \begin{pmatrix} 0\\ 1\\ 1 \end{pmatrix} \},\ u = \begin{pmatrix} 3\\ -1\\ 2 \end{pmatrix} $$
\bigskip

\begin{proof}

    La dimensión de $V$ es 3, entonces, $B$ es base si $v_{1}, v_{2}, v_{3}$ son linealmente independientes. \\
    $v_{1}, v_{2}, v_{3}$ son linealmente independientes si el determinante de la matriz de sus componentes es diferente de 0.
    \bigskip
    
    $$|B| = \begin{vmatrix} 1 & 0 & 0\\ 1 & 1 & 1\\ 0 & -1 & 1 \end{vmatrix} = 1 - (-1) = 2$$
    
    $$|B| = 2 \neq 0$$
    
\end{proof}

$$
\begin{pmatrix}[ccc|c] 1 & 0 & 0 & 3\\ 1 & 1 & 1 & -1\\ 0 & -1 & 1 & 2 \end{pmatrix} 
\sim 
\begin{pmatrix}[ccc|c] 1 & 0 & 0 & 3\\ 0 & 1 & 1 & -4\\ 0 & -1 & 1 & 2 \end{pmatrix} 
\sim 
\begin{pmatrix}[ccc|c] 1 & 0 & 0 & 3\\ 0 & 1 & 1 & -4\\ 0 & 0 & 2 & -2 \end{pmatrix} 
\sim
$$
$$
\begin{pmatrix}[ccc|c] 1 & 0 & 0 & 3\\ 0 & 1 & 1 & -4\\ 0 & 0 & 1 & -1 \end{pmatrix}
\sim
\begin{pmatrix}[ccc|c] 1 & 0 & 0 & 3\\ 0 & 1 & 0 & -3\\ 0 & 0 & 1 & -1 \end{pmatrix}
$$

\bigskip

Entonces, las coordenadas de $[u]_{B}$ son

$$\begin{pmatrix} 3\\ -3\\ -1 \end{pmatrix}$$

\bigskip

\noindent
2. Dadas las bases
$$B_{1} = \{ \hat{i}, \hat{i}-\hat{j}, \hat{i}-\hat{j}-\hat{k} \}$$
$$B_{2} = \{ 2\hat{i} + \hat{j}, 3\hat{j}, 5\hat{k} - \hat{i} \}$$
de $\mathbb{R}^3$
\bigskip

a) Halle la matriz de cambio de base de
$B_{1}$ a $B_{2}$ 
\bigskip

$$
\begin{pmatrix}[ccc|ccc] 2 & 0 & 0 & 1 & 1 & 1\\ 1 & 3 & -1 & 0 & -1 & -1\\ 0 & 0 & 5 & 0 & 0 & -1 \end{pmatrix} 
\sim
\begin{pmatrix}[ccc|ccc] 1 & 0 & 0 & \frac{1}{2} & \frac{1}{2} & \frac{1}{2}\\ 1 & 3 & -1 & 0 & -1 & -1\\ 0 & 0 & 5 & 0 & 0 & -1 \end{pmatrix} \sim
$$
$$
\begin{pmatrix}[ccc|ccc] 1 & 0 & 0 & \frac{1}{2} & \frac{1}{2} & \frac{1}{2}\\ 0 & 3 & -1 & -\frac{1}{2} & -\frac{3}{2} & -\frac{3}{2}\\ 0 & 0 & 5 & 0 & 0 & -1 \end{pmatrix} 
\sim
\begin{pmatrix}[ccc|ccc] 1 & 0 & 0 & \frac{1}{2} & \frac{1}{2} & \frac{1}{2}\\ 0 & 1 & -\frac{1}{3} & -\frac{1}{6} & -\frac{1}{2} & -\frac{1}{2}\\ 0 & 0 & 1 & 0 & 0 & -\frac{1}{5} \end{pmatrix}
\sim
$$
$$
\begin{pmatrix}[ccc|ccc] 1 & 0 & 0 & \frac{1}{2} & \frac{1}{2} & \frac{1}{2}\\ 0 & 1 & 0 & -\frac{1}{6} & -\frac{1}{2} & -\frac{17}{30}\\ 0 & 0 & 1 & 0 & 0 & -\frac{1}{5} \end{pmatrix}
$$
\bigskip

b) Encuentre las coordenadas del vector
$\Vec{w} = -\hat{i} + 2\hat{j} - 3\hat{k}$ en la base $B_{2}$, haciéndolo directamente y usando la matriz de cambio de base
\bigskip

$$\begin{pmatrix}[ccc|c] 2 & 0 & 0 & -1\\ 1 & 3 & -1 & 2\\ 0 & 0 & 5 & -3 \end{pmatrix} 
\sim
\begin{pmatrix}[ccc|c] 1 & 0 & 0 & -\frac{1}{2}\\ 0 & 3 & -1 & \frac{5}{2}\\ 0 & 0 & 5 & -3 \end{pmatrix}
\sim
\begin{pmatrix}[ccc|c] 1 & 0 & 0 & -\frac{1}{2}\\ 0 & 1 & -\frac{1}{3} & \frac{5}{6}\\ 0 & 0 & 1 & -\frac{3}{5} \end{pmatrix}
$$
$$
\begin{pmatrix}[ccc|c] 1 & 0 & 0 & -\frac{1}{2}\\ 0 & 1 & 0 & \frac{19}{30}\\ 0 & 0 & 1 & -\frac{3}{5} \end{pmatrix}
$$

\bigskip
Usando la matriz de cambio de base:
\bigskip
$$B = \begin{pmatrix} 2 & 0 & 0 \\ 1 & 3 & -1\\ 0 & 0 & 5 \end{pmatrix}$$
\bigskip
La matriz de cambio de base ($M_{B \leftarrow C}$) será $B^{-1}$, $B^{-1} = \frac{1}{|B|}Adj(B)$, donde $Adj(B) = C^{T}$.
\bigskip

$$C = \begin{pmatrix} 15 & -5 & 0\\ 0 & 10 & 0\\ 0 & 2 & 6 \end{pmatrix} \implies Adj(B) = \begin{pmatrix} 15 & 0 & 0\\ -5 & 10 & 2\\ 0 & 0 & 6 \end{pmatrix} \ \land \ |B| = 30 \ \therefore$$

$$B^{-1} = \frac{1}{30}\begin{pmatrix} 15 & 0 & 0\\ -5 & 10 & 2\\ 0 & 0 & 6 \end{pmatrix}$$

$\therefore$

$$[w]_{B} = \frac{1}{30} \begin{pmatrix} 15 & 0 & 0\\ -5 & 10 & 2\\ 0 & 0 & 6 \end{pmatrix} \begin{pmatrix} -1\\ 2\\ -3 \end{pmatrix} = \frac{1}{30} \begin{pmatrix} -15\\ 19\\ -18 \end{pmatrix}
$$

\bigskip



\noindent
3. Dada la matriz

$$A = \begin{pmatrix}
    1 & -1 & 2 & 3 & 0 \\
    -1 & 0 & -4 & 3 & 1  \\
    2 & 1 & 6 & 0 & 1 \\
    -1 & 2 & 0 & 1 & 1
    \end{pmatrix}$$
    
a) Determine una base para el espacio nulo de la matriz
\bigskip

b) Determine una base para la imagen de la matriz
\bigskip

c) Determine el rango y la nulidad de A
\bigskip

\noindent
4. Diga si las siguientes afirmaciones son verdaderas o falsas (Justifique).
\bigskip

a) Suponga una matriz de tamaño $17 \times 9$, el rango de la matriz es 10 y su nulidad es 7
\bigskip

b) Suponga $V$ el espacio vectorial de todas las matrices simétricas de tamaño $3 \times 3$ entonces su dimensión es 5
\bigskip

Falso
\begin{proof}
$$\begin{pmatrix} 0 & a & b\\ a & 0 & c\\ b & c & 0 \end{pmatrix} = a\begin{pmatrix} 0 & 1 & 0\\ 1 & 0 & 0\\ 0 & 0 & 0 \end{pmatrix} + b\begin{pmatrix} 0 & 0 & 1\\ 0 & 0 & 0\\ 1 & 0 & 0 \end{pmatrix} + c\begin{pmatrix} 0 & 0 & 0\\ 0 & 0 & 1\\ 0 & 1 & 0 \end{pmatrix}$$

$$dim(V) = 3$$
\end{proof}
\bigskip

c) Si $v_{1}; v_{2}$ es base de $V$, entonces el conjunto $v_{1} + v_{2}; v_{1} - v_{2}$ también es base de $V$.
\bigskip

Verdadero
\begin{proof}
Ya que $v_{1}$ y $v_{2}$ son base de V, podemos decir que $dim(V) = 2$ y además, $v_{1}$ y $v_{2}$ son linealmente independientes. Como $v_{1}$ y $v_{2}$ son linealmente independientes, el determinante de la matriz de sus componentes es diferente de 0, es decir

$$a_{1}b_{2} - a_{2}b_{1} \neq 0$$
\bigskip

Donde $v_{1} = \begin{pmatrix} a_{1}\\ b_{1} \end{pmatrix}$ y $v_{2} = \begin{pmatrix} a_{2}\\ b_{2} \end{pmatrix}$

$\{ v_{1} + v_{2}; v_{1} - v_{2} \}$ será una base de $V$ si $v_{1} + v_{2}$ y $v_{1} - v_{2}$ son linealmente independientes $\therefore$

$$\begin{vmatrix} a_{1} + a_{2} & a_{1} - a_{2}\\  b_{1} + b_{2} & b_{1} - b_{2} \end{vmatrix} = (a_{1} + a_{2})(b_{1} - b_{2}) - (b_{1} + b_{2})(a_{1} - a_{2})$$

$$ = a_{1}b_{1} - a_{1}b_{2} + a_{2}b_{1} - a_{2}b_{2} - (a_{1}b_{1} + a_{1}b_{2} - a_{2}b_{1} - a_{2}b_{2})$$

$$ = -2(a_{1}b_{2} - a_{2}b_{1})$$

$$a_{1}b_{2} - a_{2}b_{1} \neq 0 \ \therefore$$

$$-2(a_{1}b_{2} - a_{2}b_{1}) \neq 0$$

\end{proof}
\bigskip

d) $(0,0); (1,3)$ es una base de $\mathbb{R}^2$
\bigskip

Falso
\begin{proof}
$\begin{pmatrix} 0\\ 0 \end{pmatrix}$ y $\begin{pmatrix} 1\\ 3 \end{pmatrix}$ serán base de $\mathbb{R}^2$ si son linealmente independientes, y esto es así si el determinante de la matriz de sus componentes es diferente de 0 $\therefore$

$$\begin{vmatrix} 0 & 1\\ 0 & 3 \end{vmatrix} = 0$$

\end{proof}

\end{document}
