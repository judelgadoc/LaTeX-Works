\documentclass[11pt,letterpaper]{article}
\usepackage[utf8]{inputenc}
\usepackage[spanish]{babel}
\author{Juan Delgado}
\title{Registro de lectura: Título de la obra}

\begin{document}
\maketitle

\section{Propósito}
\textit{Para qué leo esto}

\section{Inspeccionar}

\subsection{Autor}
\textit{¿A quién estás leyendo? Realiza una corta biografía del autor, haciendo énfasis en cualquier parte de su vida que sea relevante para el texto.}

\subsection{Contexto}
\textit{Tal vez las condiciones del pasado ya no se aplican en la actualidad. Describe como era el mundo, que ha cambiado desde entonces, y porqué ha cambiado.}

\subsection{Estructura de la obra}
\textit{Menciona si la obra tiene resumen, introducción, preguntas, glosario, sumario, imágenes, figuras, tablas, etc. y describe cada una de ellas.}

\subsection{Secciones}
\begin{enumerate}
	\item 
	\item
\end{enumerate}

\section{Preguntar y predecir}
\textit{Formule un par de preguntas teniendo en cuenta el propósito y las secciones o capítulos del documento.}

\begin{enumerate}
	\item 
	\item
\end{enumerate}

\textit{Después de leer, responda brevemente cada una de las preguntas planteadas anteriormente.}

\begin{enumerate}
	\item 
	\item
\end{enumerate}

\section{Resumen breve de cada sección}
\subsection{Sección 1}
Resumen

\subsection{Sección 2}
Resumen

\section{Ideas fuerza}
\textit{Las ideas que consideras más importantes, tanto para comunicar el texto como para entenderlo.}

\subsection{Nombre de la idea 1}
Resumen

\subsection{Nombre de la idea 2}
Resumen

\section{Aplicar o transferir}
\textit{En no más de cinco renglones (preferiblemente), formule una recomendación sobre la posible utilidad de esta lectura para su trabajo como ingeniero de sistemas y computación.}

\section{Comunicar}
\textit{¿Cómo explicarías la obra a alguien que no la ha leído?}

\section{Referencias}

\end{document}
